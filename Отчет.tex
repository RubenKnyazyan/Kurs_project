%% -*- coding: utf-8 -*-
\documentclass[12pt,a4paper]{scrartcl} 
%\usepackage[14pt]{extsizes}
\usepackage[utf8]{inputenc}
\usepackage[english,russian]{babel}
\usepackage{indentfirst}
\usepackage{misccorr}
\usepackage{graphicx}
\usepackage{amsmath}
\usepackage[left=20mm, top=15mm, right=15mm, bottom=15mm, nohead, footskip=10mm]{geometry}
\begin{document}
\begin{titlepage}
  \begin{center}
    \large
    МИНИСТЕРСТВО ОБРАЗОВАНИЯ И НАУКИ \\РОССИЙСКОЙ ФЕДЕРАЦИИ\\
    федеральное государственное автономное образовательное учреждение высшего образования\\
    «САНКТ-ПЕТЕРБУРГСКИЙ ГОСУДАРСТВЕННЫЙ УНИВЕРСИТЕТ 
    АЭРОКОСМИЧЕСКОГО ПРИБОРОСТРОЕНИЯ»

    Кафедра 43   
    \end{center}
	\vfill
	\noindent КУРСОВОЙ ПРОЕКТ
    \normalsize{}\\
    \normalsize{ЗАЩИЩЕН С ОЦЕНКОЙ}\\
    \normalsize{РУКОВОДИТЕЛЬ}
    
    \underline{ст.преп}
    \hspace{5cm}
    \underline{\hspace{4cm}}
    \hspace{4cm}
    \underline{М.Д.Поляк}
    \vfill
    
	\begin{center}
	\normalsize{ПОЯСНИТЕЛЬНАЯ ЗАПИСКА}\\
	\normalsize{К КУРСОВОМУ ПРОЕКТУ}\\
	\vfill
	\normalsize{Резервное копирование}\\
	\vfill
    \textsc{по дисциплине: ОПЕРАЦИОННЫЕ СИСТЕМЫ}\\
	\end{center}

	\vfill
	\noindent РАБОТУ ВЫПОЛНИЛ
	\normalsize{}\\  
	\normalsize{СТУДЕНТК ГР.}\hspace{1cm}\underline{4331}
	\hspace{2cm}
	\underline{\hspace{3cm}}
	\hspace{3cm}
	\underline{Князян.Р.А.}
\vfill

\begin{center}
  Санкт-Петербург, 2016 г.
\end{center}
\end{titlepage}
\newpage
%\normalsize{Цель работы:реализовать демон для flash накопителей, осуществляющий резервное копирование данных под ОС Linux}\\
\tableofcontents % Вывод содержания
\newpage
\section{Цель работы}
	\normalsize{Цель работы: реализовать демон для flash накопителей, осуществляющий резервное копирование данных под ОС Linux}
\section{Задание}
	Реализовать демон для flash-накопителя, работающего через интерфейс USB, реализующий  копирование файлов между двумя flash-накопителями. При подключении двух flash-накопителей с заданными серийными номерами, демон должен запускать копирование файлов с одного накопителя на другой и сохранять текстовый лог успешно завершенных операций копирования на обоих flash-накопителях.
\section{Техническая документация}
\subsection{Установка}
	Склонировать репозиторий с github при помощи команды: \begin{verbatim}
	git clone https://github.com/RubenKnyazyan/Kurs_project.git
	\end{verbatim}Для работы демона необходима версия ядра не менее 3.19.
\subsection{Использование}
	\begin{enumerate}
	\item Сборка проекта: \\
		Необходимо перейти в корневой каталог репозитория и вызвать команду
		make. Демон будет скомпилирован.
	\item Запуск проекта: \\
		После того, как демон был скомпилирован, он будет и автоматически запущен
	\item Использование демона: \\
		Перед использованием демона необходимо указать серийные номера флеш-накопителей, путь к каталогу флеш-накопителей.
		Задаются они в файле deamon.c, (пример)\begin{verbatim}
				//указываем серийный номер накопителей
				serial1 = (char*)"07AB1608151B5B72";
				serial2 = (char*)"c3f41ad10ff131";
				dir_serial1 = (char*)"/media/ruben/STORE/";
				dir_serial2 = (char*)"/media/ruben/B480-C68C/";;
				\end{verbatim}
		После того, как мы подключили накопитель, в log-файле появится запись о том, что демон начал свою работу.
		Так же, мы увидим, что в log'e отображаются все скопированные на диск файлы.
	\item Выключение и удаление проекта: \\
		Чтобы выключить демона нам понадобиться диспетчер задач, к примеру htop. Запускаем htop, нажимаем F4, выбираем command, далее вводим ./deamon.o, и мы увидим запущенный процесс нашего демона. Нажимая F9 нам будет предложено "убить" 			процесс, соглашаемся и жмем Enter.
		Удаление исполняемого файла демона можно произвести вызвав команду make clean.
		в корневой директории репозитория.
	\end{enumerate}

\section{Выводы}
	В процессе выполнения данной курсовой работы мною были получены знания и навыки, необходимые для работы с USB носителями, потоками, файлами и папками,  в ОС семейства Linux, а так же знания и навыки в написании демонов.
	\newpage
\section{Приложения}
deamon.c:\begin{verbatim}
#include <stdio.h>
#include <stdlib.h>
#include <string.h>
#include <sys/stat.h>
#include <sys/types.h>
#include <sys/time.h>
#include <unistd.h>
#include <errno.h>
#include <fcntl.h>
#include <syslog.h>
#include <fstream>  // to copy files
#include <iostream>
#include <dirent.h> //for searching files
#include <sys/stat.h> // для создания папки
#include <sys/types.h> // для создания папки



static char* serial1 = (char*)"07AB1608151B5B72";
static char* dir_serial1 = (char*)"/media/ruben/STORE/";
static char* serial2 = (char*)"c3f41ad10ff131";
static char* dir_serial2 = (char*)"/media/ruben/B480-C68C/";
//static char* dir_comp = (char*)"/home/ruben/try2/1/";
static char* path_log_file1 = (char*)"/media/ruben/STORE/1.log";
static char* path_log_file2 = (char*)"/media/ruben/B480-C68C/1.log";

int Deamon(void);
int writeLog(char msg[256]);
char* getSerial( char *str );
int isSerial(char* ser);
void copyFile(std::string pathFrom, std::string pathTo, std::string fileName);
void copyDir(std::string pathFrom, std::string pathTo);
int fileSize( std::fstream &f );

//функция получающая серийный номер
char* getSerial( char *str ) 
{
    ssize_t len;
    char buf[256];
    char *p;
    char buf2[256];
    int i;
    static char comText[256];
    bzero(comText, 256);
    strcpy(comText, "");

    len = readlink(str, buf, 256);
    if (len <= 0) {
        return (char*)"-";
    }
    buf[len] = '\0';
    sprintf(buf2, "%s/%s", "/sys/block/", buf);
    for (i=0; i<6; i++) {
        p = strrchr(buf2, '/');
        *p = 0;
    }
    strcat(buf2, "/serial");

    int f = open(buf2, 0);
    if (f == -1) return (char*)"-";
    len = read(f, buf, 256);
    if (len <= 0) {
        return (char*)"-";
    }
    buf[len-1] = '\0';
    
    strcat(comText, buf);

    return comText;
}

//функция проверяющая серийный номер на соответствие заданному
int isSerial(char* ser)
{
    if (strcmp(getSerial((char*)"/sys/block/sdb"), ser) == 0
        || strcmp(getSerial((char*)"/sys/block/sdc"), ser) == 0
        || strcmp(getSerial((char*)"/sys/block/sdd"), ser) == 0
        || strcmp(getSerial((char*)"/sys/block/sdf"), ser) == 0
        || strcmp(getSerial((char*)"/sys/block/sdg"), ser) == 0)
    {
        return 1;
    }
    return 0;
}


int writeLog(char msg[256]) { //функция записи строки в лог

    FILE * pLog;
    pLog = fopen(path_log_file1, "a");
    if(pLog == NULL) {
        return 1;
    }
    char str[1024];
    bzero(str, 1024);
    strcat(str, msg);
    strcat(str, (char*)"\n");
    fputs(str, pLog);
    fclose(pLog);

	FILE * pLog2;
    pLog2 = fopen(path_log_file2, "a");
    if(pLog2 == NULL) {
        return 1;
    }
    char str2[1024];
    bzero(str2, 1024);
    strcat(str2, msg);
    strcat(str2, (char*)"\n");
    fputs(str2, pLog2);
    fclose(pLog2);
    return 0;
}


void copyFile(std::string pathFrom, std::string pathTo, std::string fileName)
{
    writeLog((char*) "Copping...");

    std::string _pathFrom = pathFrom;
    std::string _pathTo = pathTo;

    _pathFrom = _pathFrom+fileName;
    _pathTo = _pathTo+fileName;

    char myA[fileName.size()+1];
    strcpy(myA, fileName.c_str());

    if (strcmp(myA,".") == 0 || strcmp(myA,"..") == 0)
    {
        writeLog((char*) "New Branch");
        
    }
    else
    {
        const char * c1 = _pathFrom.c_str(); 
        const char * c2 = _pathTo.c_str(); 

        char * buffer;
        buffer = new char;
    
        std::fstream infile(c1, std::ios::in | std::ios::binary);
        if (!infile.is_open()) {// если файл не открыт
           writeLog((char*) "Файл не открывается!"); // сообщить об этом
           //writeLog((char*) c1);
           //
       }
        else 
        {
            writeLog((char*) "Файл открылся!");
            infile.read(buffer, sizeof(char)); //читаем первый сивол
            std::string log = "";

            if (infile.fail() && !infile.eof()) //проверка на папку
            {
                infile.close();
                remove(_pathTo.c_str());
                log = c2;
                log = "dir: " + log;
                mkdir(_pathTo.c_str(), 0755);
                //writeLog((char*) _pathTo);
                writeLog((char*)" -- Это папка.\n");
                copyDir(_pathFrom + "/", _pathTo + "/");
            }
            else 
            { //иначе файл
                std::fstream outfile(c2, std::ios::out | std::ios::app | std::ios::binary); //открываем файл для записи  
                outfile.seekg(0, std::ios::beg);

                if (fileSize(outfile) != fileSize(infile) && fileSize(infile) > 0) //если размеры == то не перезаписываем и не пустой
                {
                    outfile.close();
                    outfile.open(c2, std::ios::out | std::ios::binary);
                    log = c2;
                    log = "new file: " + log;
                    while (!infile.eof() && infile.good())
                    {
                        outfile.write(buffer, sizeof(char));
                        infile.read(buffer, sizeof(char));
                    }
                } else 
                {
                    log = c2;
                    log = "file exists: " + log;
                }
                infile.close();
                outfile.close();
            }
            writeLog((char *) log.c_str());
        }
        delete buffer;
        std::cout << _pathFrom << "->" << _pathTo << std::endl;
    }
}

void copyDir(std::string pathFrom, std::string pathTo)
{
     writeLog((char*) "Папка тут!");
    struct dirent **namelist;
    int n;

    n = scandir(pathFrom.c_str(), &namelist, 0, alphasort);
    if (n >= 0)
    {
        while (n--) 
        {
            copyFile(pathFrom, pathTo, namelist[n]->d_name);           
            free(namelist[n]);
        }
        free(namelist);
    }
}


int fileSize( std::fstream &f )
{
    int size, saveTellg;
    saveTellg = f.tellg();
    f.seekg(0, std::ios::end);
    size = f.tellg();
    f.seekg(saveTellg, std::ios::beg);
    return size;
}

int Deamon(void) { //наш бесконечный цикл демона
    while(1) {
    if(isSerial(serial1) && isSerial(serial2)) // Если это необходимая флешка
        {
            writeLog((char*) dir_serial1);
			writeLog((char*) "  to  ");
			writeLog((char*) dir_serial2);
            copyDir(dir_serial1, dir_serial2);//запускаем функцию копирования
			//copyDir(dir_serial2, dir_serial1);//запускаем функцию копирования
            writeLog((char*)"\n+++++++++++++++++++++++++++++++++++++++\n");
        }
        sleep(3);//ждем до следующей итерации
    }
    return 0;
}

int main(int argc, char* argv[]) {
    writeLog((char*)"USBDeamon Start");

    pid_t parpid, sid;
    
    parpid = fork(); //создаем дочерний процесс

    if(parpid < 0) 
        {
            exit(1);
        } 
    else if(parpid != 0) 
        {
            exit(0);
        } 

    umask(0);//даем права на работу с фс

    sid = setsid();//генерируем уникальный индекс процесса

    if(sid < 0) 
        {
            exit(1);
        }

    if((chdir("/")) < 0) 
        {//выходим в корень фс
            exit(1);
        }

    close(STDIN_FILENO);//закрываем доступ к стандартным потокам ввода-вывода
    close(STDOUT_FILENO);
    close(STDERR_FILENO);
    
    return Deamon();
}

\end{verbatim}
	\newpage
\end{document}
