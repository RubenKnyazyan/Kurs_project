%% -*- coding: utf-8 -*-
\documentclass[12pt,a4paper]{scrartcl} 
%\usepackage[14pt]{extsizes}
\usepackage[utf8]{inputenc}
\usepackage[english,russian]{babel}
\usepackage{indentfirst}
\usepackage{misccorr}
\usepackage{graphicx}
\usepackage{amsmath}
\usepackage[left=20mm, top=15mm, right=15mm, bottom=15mm, nohead, footskip=10mm]{geometry}
\begin{document}
\begin{titlepage}
  \begin{center}
    \large
    МИНИСТЕРСТВО ОБРАЗОВАНИЯ И НАУКИ \\РОССИЙСКОЙ ФЕДЕРАЦИИ\\
    федеральное государственное автономное образовательное учреждение высшего образования\\
    «САНКТ-ПЕТЕРБУРГСКИЙ ГОСУДАРСТВЕННЫЙ УНИВЕРСИТЕТ 
    АЭРОКОСМИЧЕСКОГО ПРИБОРОСТРОЕНИЯ»

    Кафедра 43   
    \end{center}
	\vfill
	\noindent КУРСОВОЙ ПРОЕКТ
    \normalsize{}\\
    \normalsize{ЗАЩИЩЕН С ОЦЕНКОЙ}\\
    \normalsize{РУКОВОДИТЕЛЬ}
    
    \underline{ст.преп}
    \hspace{5cm}
    \underline{\hspace{4cm}}
    \hspace{4cm}
    \underline{М.Д.Поляк}
    \vfill
    
	\begin{center}
	\normalsize{ПОЯСНИТЕЛЬНАЯ ЗАПИСКА}\\
	\normalsize{К КУРСОВОМУ ПРОЕКТУ}\\
	\vfill
	\normalsize{Резервное копирование}\\
	\vfill
    \textsc{по дисциплине: ОПЕРАЦИОННЫЕ СИСТЕМЫ}\\
	\end{center}

	\vfill
	\noindent РАБОТУ ВЫПОЛНИЛ
	\normalsize{}\\  
	\normalsize{СТУДЕНТК ГР.}\hspace{1cm}\underline{4331}
	\hspace{2cm}
	\underline{\hspace{3cm}}
	\hspace{3cm}
	\underline{Князян.Р.А.}
\vfill

\begin{center}
  Санкт-Петербург, 2016 г.
\end{center}
\end{titlepage}
\newpage
%\normalsize{Цель работы:реализовать демон для flash накопителей, осуществляющий резервное копирование данных под ОС Linux}\\
\tableofcontents % Вывод содержания
\newpage
\section{Цель работы}
	\normalsize{Цель работы: реализовать демон для flash накопителей, осуществляющий резервное копирование данных под ОС Linux}
\section{Задание}
	Реализовать демон для flash-накопителя, работающего через интерфейс USB, реализующий  копирование файлов между двумя flash-накопителями. При подключении двух flash-накопителей с заданными серийными номерами, демон должен запускать копирование файлов с одного накопителя на другой и сохранять текстовый лог успешно завершенных операций копирования на обоих flash-накопителях.
\section{Техническая документация}
\subsection{Установка}
	Склонировать репозиторий с github при помощи команды: \begin{verbatim}
	git clone https://github.com/RubenKnyazyan/Kurs_project.git
	\end{verbatim}Для работы демона необходим установленный python версии 3.5.2 и выше.
\subsection{Использование}
	Разработанный демон поддерживает следующие команды: start, stop, status. Вызов команды осуществляется следующим образом: python deamon.py <команда>. Команда start запускает демона и создает PID-файл с ID процесса. Команда status сообщает о статусе демона(работает он или нет), а так же ID процесса, если демон запущен. Команда stop останавливает работу демона.  Команда restart перезапускает демон. 

\section{Выводы}
	В процессе выполнения данной курсовой работы мною были получены знания и навыки, необходимые для работы с USB носителями, потоками, файлами и папками,  в ОС семейства Linux, а так же знания и навыки в написании демонов.
	\newpage
\section{Приложения}
deamon.ру:\begin{verbatim}
#!/usr/bin/env python3
import sys
import os
import shutil
import logging
import time
import subprocess
import pyudev
from daemonize import Daemon
 
context = pyudev.Context()
logging.basicConfig(filename='/tmp/kursovoi.log', level=logging.DEBUG)
SOURCE_SERIAL = 'Verbatim_STORE_N_GO_07AB1608151B5B72'
DESTINATION_SERIAL = 'Corsair_Voyager_Mini_c3f41ad10ff131'
WAIT = 60
PIDFILE = '/tmp/daemon-kurs.pid'
 
class KursDaemon(Daemon):
 
    def run(self):
        while True:
            if try_copy():
                sys.exit(0)
            else:
                time.sleep(WAIT)
 
 
def copytree(src, dst):
    result = None
    process = subprocess.Popen('cp -rvf {0} {1}'.format(src, dst).split(), stdout=subprocess.PIPE)
    result, error = process.communicate()
    logging.debug(result)
    return result.decode('utf-8').split('\n')
 
 
def try_copy():
    src, dst = None, None
 
    # получить device name для флешек
    src_id = '/dev/disk/by-id/usb-{0}-0:0-part1'.format(SOURCE_SERIAL)
    dst_id = '/dev/disk/by-id/usb-{0}-0:0-part1'.format(DESTINATION_SERIAL)
    logging.debug(src_id)
    src_device_name, dst_device_name = None, None
    if os.path.islink(src_id):
        src_device_name = os.readlink(src_id).split("/")[-1]
        logging.debug(src_device_name)
    if os.path.islink(dst_id):
        dst_device_name = os.readlink(dst_id).split("/")[-1]
    if not src_device_name:
        logging.error('No source device')
        return False
    if not dst_device_name:
        logging.error('No destination device')
        return False
    # получить пути, куда смонтированы
    mtab = open('/etc/mtab', 'r')
    src_path, dst_path = None, None
    for line in mtab.readlines():
        if src_device_name in line:
            src_path = line.split(' ')[1]
        if dst_device_name in line:
            dst_path = line.split(' ')[1]
    if not src_path:
        logging.error('Source device is not mounted!')
    if not dst_path:
        logging.error('Destination device is not mounted!')
    if not src_path or not dst_path:
        return False
    # собственно копирование
    result = copytree(src_path + '/', dst_path + '/')
    dst_log = open(dst_path + '/copied.log', 'w')
    src_log = open(src_path + '/copied.log', 'w')
    for f in result:
        dst_log.write(f + '\n')
        src_log.write(f + '\n')
    dst_log.close()
    src_log.close()
    return True
 
 
if __name__ == '__main__':
    if len(sys.argv) != 2 or sys.argv[1] not in ["start", "stop", "status", "restart"]:
        print("Запускайте так: python main.py start|stop|status|restart")
        sys.exit(1)
    daemon = KursDaemon(PIDFILE)
    if sys.argv[1] == "start":
        daemon.start()
    elif sys.argv[1] == "stop":
        daemon.stop()
    elif sys.argv[1] == "restart":
        daemon.restart()
    elif sys.argv[1] == "status":
        if os.path.isfile(PIDFILE):
            pidfile = open(PIDFILE, "r")
            pid = pidfile.read()
            print("Запущен с PID={0}".format(pid))
        else:
            print("Демон пока что не запущен")
Class Deamon
import sys, os, time, atexit
from signal import SIGTERM

class Daemon:

        def __init__(self, pidfile, stdin='/dev/null', stdout='/dev/null', stderr='/dev/null'):
                self.stdin = stdin
                self.stdout = stdout
                self.stderr = stderr
                self.pidfile = pidfile

        def daemonize(self):
                try:
                        pid = os.fork()
                        if pid > 0:
                                # exit first parent
                                sys.exit(0)
                except OSError as e:
                        sys.stderr.write("fork #1 failed: %d (%s)\n" % (e.errno, e.strerror))
                        sys.exit(1)

                # decouple from parent environment
                os.chdir("/")
                os.setsid()
                os.umask(0)

                # do second fork
                try:
                        pid = os.fork()
                        if pid > 0:
                                # exit from second parent
                                sys.exit(0)
                except OSError as e:
                        sys.stderr.write("fork #2 failed: %d (%s)\n" % (e.errno, e.strerror))
                        sys.exit(1)

                # redirect standard file descriptors
                sys.stdout.flush()
                sys.stderr.flush()
                si = open(self.stdin, 'r')
                so = open(self.stdout, 'a+')
                # se = open(self.stderr, 'a+', 0)
                os.dup2(si.fileno(), sys.stdin.fileno())
                os.dup2(so.fileno(), sys.stdout.fileno())
                # os.dup2(se.fileno(), sys.stderr.fileno())

                # write pidfile
                atexit.register(self.delpid)
                pid = str(os.getpid())
                open(self.pidfile,'w+').write("%s\n" % pid)

        def delpid(self):
                os.remove(self.pidfile)

        def start(self):
                """
                Start the daemon
                """
                # Check for a pidfile to see if the daemon already runs
                try:
                        pf = open(self.pidfile,'r')
                        pid = int(pf.read().strip())
                        pf.close()
                except IOError:
                        pid = None

                if pid:
                        message = "pidfile %s already exist. Daemon already running?\n"
                        sys.stderr.write(message % self.pidfile)
                        sys.exit(1)

                # Start the daemon
                self.daemonize()
                self.run()

        def stop(self):
                """
                Stop the daemon
                """
                # Get the pid from the pidfile
                try:
                        pf = open(self.pidfile,'r')
                        pid = int(pf.read().strip())
                        pf.close()
                except IOError:
                        pid = None

                if not pid:
                        message = "pidfile %s does not exist. Daemon not running?\n"
                        sys.stderr.write(message % self.pidfile)
                        return # not an error in a restart

                # Try killing the daemon process       
                try:
                        while 1:
                                os.kill(pid, SIGTERM)
                                time.sleep(0.1)
                except OSError as err:
                        err = str(err)
                        if err.find("No such process") > 0:
                                if os.path.exists(self.pidfile):
                                        os.remove(self.pidfile)
                        else:
                                print(str(err))
                                sys.exit(1)

        def restart(self):
                """
                Restart the daemon
                """
                self.stop()
                self.start()

        def run(self):
                """
                You should override this method when you subclass Daemon. It will be called after the process has been
                daemonized by start() or restart().
                """

\end{verbatim}
	\newpage
\end{document}
